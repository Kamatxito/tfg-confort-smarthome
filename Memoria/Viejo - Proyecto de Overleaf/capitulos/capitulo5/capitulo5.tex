\chapter{Estudio de índices de confort térmico}

La recogida de los parámetros atmosféricos realizada y explicada hasta ahora, tiene una finalidad. El objetivo de recogerlos y observarlos, tanto sus valores históricos como actuales, es poder gestionar y alcanzar un confort térmico adecuado, un tema que pese a haberse definido en los últimos años, llevamos intentando alcanzar los seres humanos toda la vida.

El confort térmico tiene varias definiciones, es un concepto ambiguo, pero podríamos definirlo como la sensación correcta y agradable que tiene una persona en un entorno respecto a la temperatura. El objetivo de la gestión del confort térmico sería llegar a ese punto donde el usuario se siente agradable dentro de una estancia o al menos en un rango que se considera que está dentro del confort.

Para conocer el punto o rango de confort térmico, con los años se han diseñado y mejorado varios modelos, algunos de ellos son:

\section{Modelo PMV}

(TO-DO)

\section{Modelo Adaptativo}

(TO-DO)

Saber definir cuál es el punto o rango de esa sensación agradable es la clave para definir el confort térmico. Aunque a la hora de hablar de confort en general, no solo se debe tener en cuenta los parámetros relacionados con la temperatura, como la humedad, sino que se debe hablar de otros tipo de conforts: visual, en lo referido a la luminosidad de la estancia; y la calidad del aire, referido a la contaminación o a la cantidad de CO2 de este.

(…)

De forma natural, el ser humano ha intentado a lo largo de la historia construir sus casas teniendo en cuenta la orientación del sol, el viento que corre en una zona específica, descubriendo qué materiales son más aislantes que otros o la forma en la que construyen sus casas para poder hacerlas lo más confortables posibles.