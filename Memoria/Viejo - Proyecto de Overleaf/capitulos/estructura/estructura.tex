{\huge \textbf{Prefacio}}

%Espacio de separación
\vspace{1cm}

El objetivo principal de este proyecto es implementar o desarrollar un prototipo de frigorífico inteligente, denominado SmartFridge. Para ello, se ha realizado una memoria explicando todo el sistema, que servirá para entender todo el proceso de la creación de éste, además de aprender todas las acciones que el sistema tiene y como se utilizan. La estructura a seguir en esta memoria es la siguiente:

En el primer capítulo se realiza una introducción al tema del trabajo realizado. Además, se describen los lenguajes de programación que se han utilizado para realizar el proyecto, las formas de comunicación con los tutores, los entornos de desarrollo que se han usado, el software usado para la documentación y por último las tecnologías que se han usado para realizar el proyecto.

En cuanto al capítulo 2, se especifica el objetivo principal y los subobjetivos del proyecto, además de las tareas a realizar a lo largo de éste para cumplir todos los objetivos especificados.

En el capítulo 3, se realiza una descripción de las fases en las cuales se ha dividido el proyecto. Además, se muestran varios diagramas en los que se puede apreciar la diferencia entre la planificación del proyecto y la ejecución de éste.

En el capítulo 4, se muestra la capa de dispositivos físicos, con la explicación y comparación de cada uno, se decide cual de ellos usar y se presenta la localización de cada uno de ellos y el presupuesto.

En el capítulo 5, se muestra la capa de neblina o niebla, donde se presentan los controladores y los diferentes protocolos de comunicación disponible y se elige el más adecuado para esté trabajo. Además, se muestra la localización y presupuesto de estos y el funcionamiento del código usado de forma breve.

En el capítulo 6, se muestra la capa de red en la que se explica cómo es la conexión entre el controlador con los sensores, y la base de datos de datos del sistema. Además, se explica el funcionamiento del código brevemente.

En el capítulo 7, se muestra la capa de la nube en la que se explica los requisitos funcionales y no funcionales del sistema. Además de eso, se realiza una explicación breve de toda la estructura del código y su funcionamiento.

En el siguiente capítulo, el capítulo 8, se muestra el software utilizado para hacer los modelos 3D necesarios, los modelos en sí, y el proceso para ser instalados en el frigorífico.

El capítulo 9 muestra la simulación de uso de las opciones mas generales del sistema, con fotografías de la interfaz y un vídeo para verlo mas detalladamente.

Finalmente, tenemos un capítulo 10 para el resultado y la conclusión del proyecto y posibles trabajos futuros. Además, tenemos una bibliografía y un anexo.