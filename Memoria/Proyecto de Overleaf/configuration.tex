%Formato libro y con tamaño de letra a 11
\documentclass[11pt]{book} 
%Configurar el idioma en español 
\usepackage[spanish, es-tabla]{babel} 
\usepackage[utf8]{inputenc}
%Margenes de la hoja
\usepackage[top=2.5cm,bottom=2.5cm,left=2.5cm,right=2.5cm]{geometry} 
%Crea separación entre dos lineas (interlineado)
\usepackage{setspace} 
\spacing{1}
%Para insertar imágenes
\usepackage{graphicx} 
%Espaciado entre párrafos
\setlength{\parskip}{8px} 
%Para crear los headers y footers
\usepackage{fancyhdr} 
%Para que las figuras salgan con subnumeros
\usepackage{amsmath}
\numberwithin{figure}{section}
\numberwithin{table}{section}
%Para incluir PDFs
\usepackage{pdfpages} 
%Para poder poner mas subsecciones de las que por defecto vienen predefinidas
\makeatletter
\renewcommand\paragraph{\@startsection{paragraph}{4}{\z@}%
            {-2.5ex\@plus -1ex \@minus -.25ex}%
            {1.25ex \@plus .25ex}%
            {\normalfont\normalsize\bfseries}}
\makeatother
\setcounter{secnumdepth}{4} 
\setcounter{tocdepth}{4}
%Para poder poner las imágenes rotadas con el label y todo.
\usepackage{adjustbox}
%Para la bibliografia
\usepackage[sorting=none]{biblatex}
\addbibresource{bibliografia.bib}
% Para hacer que las imagenes esten al final de una hoja
\usepackage{dblfloatfix}    
%Para hacer itemize con dos columnas.
\usepackage{multicol}
%Cambiar tamaño entre las filas de una tabla
\renewcommand{\arraystretch}{1.2}
%Para usar subimagenes
\usepackage{caption}
\usepackage{subcaption}
%Para usar tablas que se puedan dividir en diferentes hojas
\usepackage{longtable}
%Definir nombre de TFG
\newcommand{\nombreTFG}{Reconocimiento de acciones en un frigorífico y toma de decisiones en un ambiente inteligente}
%Reajusta los números de las figuras y de las tablas
\usepackage{chngcntr}
\counterwithout{figure}{section}
\counterwithout{table}{section}
\renewcommand\thefigure{\thechapter.\arabic{figure}}
\renewcommand\thetable{\thechapter.\arabic{table}}
%Poner el titulo del capítulo en una línea.
\usepackage{titlesec}
\titleformat{\chapter}[hang] 
{\normalfont\huge\bfseries}{\chaptertitlename\ \thechapter:}{1em}{} 
%Imagenes a la derecha en las herramientas
\usepackage{wrapfig}
%Colores de las tablas
\usepackage{colortbl}
\newcommand{\gray}{\rowcolor[gray]{.90}}
\usepackage{lscape}